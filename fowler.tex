\documentclass[14pt,aspectratio=169]{beamer}

\usepackage{xcolor}
\usepackage{colortbl}
\usepackage{pgf}
\usepackage{amsmath}
\usepackage{amssymb}
\usepackage{latexsym}
\usepackage{tikz}
\usepackage{pgfplots}
\usepackage{pdfpages}
\usepackage{ulem}
\usepackage{ccicons}

\newcommand{\hsc}[1]{{\footnotesize\MakeUppercase{#1}}}

\newcounter{multiplechoice}
\renewenvironment{problem}[1]{\setcounter{multiplechoice}{0}\textcolor{gray!50!white}{\textbf{\scalebox{0.7}{PROBLEM} #1}} \\[0.5ex]}{}

\newcommand{\choice}[1]{\stepcounter{multiplechoice}\textcolor{gray}{\fbox{\scalebox{0.8}{\makebox[\widthof{\textbf{W}}]{\textbf{\Alph{multiplechoice}}}}}} #1 \\[0.25ex]}

% #1 is pre, #2 is post, and #3-#4 is the 95% CI
\newcommand{\prepost}[4]{\null\hfill\makebox[0in][r]{\textcolor{red!58!white}{{Pretest} #1\%}\textcolor{black!58!white}{\ldots}\textcolor{blue!58!white}{Posttest #2\%}}\vspace{-18pt}}


\usepackage{rotating}

\usetikzlibrary{positioning, shapes}
\usetikzlibrary{decorations.pathmorphing}

\definecolor{shaded}{RGB}{70,60,255}
\usecolortheme[named=shaded]{structure}

\definecolor{stressed}{RGB}{150,40,40}
\setbeamercolor{alerted_text}{fg=stressed}

\setbeamertemplate{navigation symbols}{}
\setbeamersize{text margin left=3mm} 
\setbeamersize{text margin right=3mm} 

\setbeamertemplate{sidebar right}{default}{}

\makeatletter
\define@key{beamerframe}{nofills}[true]{% top
  \beamer@frametopskip=0pt\relax%
  \beamer@framebottomskip=0pt\relax%
  \beamer@frametopskipautobreak=\beamer@frametopskip\relax%
  \beamer@framebottomskipautobreak=\beamer@framebottomskip\relax%
  \def\beamer@initfirstlineunskip{%
    \def\beamer@firstlineitemizeunskip{%
      \vskip-\partopsep\vskip-\topsep\vskip-\parskip%
      \global\let\beamer@firstlineitemizeunskip=\relax}%
    \everypar{\global\let\beamer@firstlineitemizeunskip=\relax}}
}
\makeatother

\newcommand{\defnword}[1]{\textbf{#1}}

\usepackage{amsthm}
\newtheorem*{conjecture}{Conjecture}

\newtheorem{observation}[theorem]{Observation}
%\newtheorem{definition}[theorem]{Definition}
\newtheorem{remark}[theorem]{Remark}
\newtheorem{claim}[theorem]{Claim}
\newtheorem{proposition}[theorem]{Proposition}
\newtheorem{question}[theorem]{Question}

\newcommand{\Z}{\mathbb{Z}}
\newcommand{\Q}{\mathbb{Q}}
\newcommand{\R}{\mathbb{R}}
\newcommand{\OO}{\mathbb{O}}
\newcommand{\HH}{\mathbb{H}}
\newcommand{\RP}{\mathbb{R}P}
\newcommand{\CP}{\mathbb{C}P}
\newcommand{\HP}{\mathbb{H}P}
\newcommand{\OP}{\mathbb{O}P}
\DeclareMathOperator{\Spin}{Spin}
\DeclareMathOperator{\Homeo}{Homeo}
\DeclareMathOperator{\SO}{SO}
\DeclareMathOperator{\fiber}{fiber}
\DeclareMathOperator{\proj}{proj}

\newcommand{\LL}{\mathbb{L}}
\newcommand{\setbackgroundblack}{%
\usebackgroundtemplate{
\begin{pgfpicture}{0in}{0in}{\paperwidth}{\paperheight}
\color{black}
\pgfrect[fill]{\pgfxy(0,0)}{\pgfpoint{\paperwidth}{\paperheight}}
\end{pgfpicture}
}
}

\newcommand{\setbackgroundgreen}{%
\usebackgroundtemplate{
\begin{pgfpicture}{0in}{0in}{\paperwidth}{\paperheight}
\color{red!50!black}
\pgfrect[fill]{\pgfxy(0,0)}{\pgfpoint{\paperwidth}{\paperheight}}
\end{pgfpicture}
}
}

\newcommand{\sectionslide}[1]{%
\setbackgroundblack
\begin{frame}[nofills]
\Huge
\vfill
\scaletowidth{\textwidth}{\textcolor{white}{\textbf{#1}}}
\vfill
\end{frame}
\clearbackgroundpicture
}

\newcommand{\encouragement}[1]{%
\setbackgroundgreen
\begin{frame}[nofills]
\Huge
\vfill
\scaletowidth{\textwidth}{\textcolor{white}{#1}}
\vfill
\end{frame}
\clearbackgroundpicture
}

\definecolor{FootColor}{rgb}{0.322,0.322,0.322}%
\definecolor{FootBackgroundColor}{rgb}{1,1,1}%

\setbeamercolor{bottomcolor}{fg=black,bg=gray!15!white}

%\setbeamertemplate{footline}{%
%\usebeamerfont{structure}
%\footnotesize
%\begin{tikzpicture}[overlay,remember picture]%
%  \node[opacity=0.8,text opacity=1,anchor=base west,yshift=2pt,xshift=-0.5mm,color=FootColor,fill=FootBackgroundColor] (website) at (current page.south west) {mooculus.osu.edu};
%  \node[opacity=0.8,text opacity=1,anchor=base east,yshift=2pt,xshift=0.5mm,color=FootColor,fill=FootBackgroundColor] (twitter) at (current page.south east) {\#mooculus};
%\end{tikzpicture}
%}

%%%%%%%%%%%%%%%%%%%%%%%%%%%%%%%%%%%%%%%%%%%%%%%%%%%%%%%%%%%%%%%%
% stack two things so that they have the same size
\newlength{\firstline}
\newlength{\secondline}
\newcommand{\stacksame}[2]{%
\setlength{\firstline}{\widthof{#1}}%
\setlength{\secondline}{\widthof{#2}}%
\pgfmathsetmacro{\myratio}{\firstline/\secondline}%
\shortstack{#1\\\scalebox{\myratio}{#2}}}

\newlength{\myscalewidth}
\newcommand{\scaletowidth}[2]{%
\setlength{\myscalewidth}{\widthof{#2}}%
\pgfmathsetmacro{\myscaleratio}{#1/\myscalewidth}%
\scalebox{\myscaleratio}{#2}}


%%%%%%%%%%%%%%%%%%%%%%%%%%%%%%%%%%%%%%%%%%%%%%%%%%%%%%%%%%%%%%%%
% I like words in front of faded images

\newcommand{\setbackgroundpicturewhite}[1]{%
\definecolor{FootColor}{rgb}{0.322,0.322,0.322}%
\definecolor{FootBackgroundColor}{rgb}{1,1,1}%
\setbeamercolor{bottomcolor}{fg=black,bg=gray!15!white}%
\usebackgroundtemplate{%
\begin{tikzpicture}[overlay,remember picture]%
\draw[fill=white] (current page.north west) rectangle (current page.south east);%
\node[fill=white,minimum width=\paperwidth,minimum height=\paperheight,yshift=1.5mm] [anchor=north west] (mynode) {\hspace{-1.5mm}\includegraphics[width=\paperwidth]{#1}};%
\end{tikzpicture}%
}}


\newcommand{\settallbackgroundpicturewhite}[1]{%
\definecolor{FootColor}{rgb}{0.322,0.322,0.322}%
\definecolor{FootBackgroundColor}{rgb}{1,1,1}%
\setbeamercolor{bottomcolor}{fg=black,bg=gray!15!white}%
\usebackgroundtemplate{%
\begin{tikzpicture}[overlay,remember picture]%
\draw[fill=white] (current page.north west) rectangle (current page.south east);%
\node[minimum width=\paperwidth,minimum height=\paperheight,yshift=1.5mm] [anchor=north west] (mynode) {\hspace{-1.5mm}\includegraphics[height=\paperheight]{#1}};%
\end{tikzpicture}%
}}


\newcommand{\settallbackgroundpictureblack}[1]{%
\definecolor{FootColor}{rgb}{0.678,0.678,0.678}%
\definecolor{FootBackgroundColor}{rgb}{0,0,0}%
\setbeamercolor{bottomcolor}{fg=black,bg=gray!15!white}%
\usebackgroundtemplate{%
\begin{tikzpicture}[overlay,remember picture]%
\draw[fill=black] (current page.north west) rectangle (current page.south east);%
\node[minimum width=\paperwidth,minimum height=\paperheight,yshift=1.5mm] [anchor=north west] (mynode) {\hspace{-1.5mm}\includegraphics[height=\paperheight]{#1}};%
\end{tikzpicture}%
}}


\newcommand{\setbackgroundpictureblack}[1]{%
\definecolor{FootColor}{rgb}{0.678,0.678,0.678}%
\definecolor{FootBackgroundColor}{rgb}{0,0,0}%
\setbeamercolor{bottomcolor}{fg=white,bg=gray!15!black}%
\usebackgroundtemplate{%
\begin{tikzpicture}[overlay,remember picture]%
\draw[fill=black] (current page.north west) rectangle (current page.south east);%
\node[minimum width=\paperwidth,minimum height=\paperheight,yshift=1.5mm] [anchor=north west] (mynode) {\hspace{-1.5mm}\includegraphics[width=\paperwidth]{#1}};%
\end{tikzpicture}%
}}


\newcommand{\setdarkbackgroundpictureblack}[1]{%
\definecolor{FootColor}{rgb}{0.678,0.678,0.678}%
\definecolor{FootBackgroundColor}{rgb}{0,0,0}%
\setbeamercolor{bottomcolor}{fg=white,bg=gray!15!black}
\usebackgroundtemplate{%
\begin{tikzpicture}[overlay,remember picture]%
\draw[fill=black] (current page.north west) rectangle (current page.south east);%
\node[minimum width=\paperwidth,minimum height=\paperheight,yshift=1.5mm] [anchor=north west] (mynode) {\hspace{-1.5mm}\includegraphics[width=\paperwidth]{#1}};%
\draw[fill=black,opacity=0.75] (current page.north west) rectangle (current page.south east);%
\end{tikzpicture}%
}}%


\newcommand{\setdarkbackgroundpicturewhite}[1]{%
\definecolor{FootColor}{rgb}{0.322,0.322,0.322}%
\definecolor{FootBackgroundColor}{rgb}{1,1,1}%
\setbeamercolor{bottomcolor}{fg=black,bg=gray!15!white}
\usebackgroundtemplate{%
\begin{tikzpicture}[overlay,remember picture]%
\draw[fill=white] (current page.north west) rectangle (current page.south east);%
\node[minimum width=\paperwidth,minimum height=\paperheight,yshift=1.5mm] [anchor=north west] (mynode) {\hspace{-1.5mm}\includegraphics[width=\paperwidth]{#1}};%
\draw[fill=white,opacity=0.75] (current page.north west) rectangle (current page.south east);%
\end{tikzpicture}%
}}%

\newcommand{\settalldarkbackgroundpicturewhite}[1]{%
\definecolor{FootColor}{rgb}{0.322,0.322,0.322}%
\definecolor{FootBackgroundColor}{rgb}{1,1,1}%
\setbeamercolor{bottomcolor}{fg=black,bg=gray!15!white}
\usebackgroundtemplate{%
\begin{tikzpicture}[overlay,remember picture]%
\draw[fill=white] (current page.north west) rectangle (current page.south east);%
\node[minimum width=\paperwidth,minimum height=\paperheight,yshift=1.5mm] [anchor=north west] (mynode) {\hspace{-1.5mm}\includegraphics[height=\paperheight]{#1}};%
\draw[fill=white,opacity=0.75] (current page.north west) rectangle (current page.south east);%
\end{tikzpicture}%
}}%


\newcommand{\settalldarkbackgroundpictureblack}[1]{%
\definecolor{FootColor}{rgb}{0.678,0.678,0.678}%
\definecolor{FootBackgroundColor}{rgb}{0,0,0}%
\setbeamercolor{bottomcolor}{fg=black,bg=gray!15!white}
\usebackgroundtemplate{%
\begin{tikzpicture}[overlay,remember picture]%
\draw[fill=black] (current page.north west) rectangle (current page.south east);%
\node[minimum width=\paperwidth,minimum height=\paperheight,yshift=1.5mm] [anchor=north west] (mynode) {\hspace{-1.5mm}\includegraphics[height=\paperheight]{#1}};%
\draw[fill=black,opacity=0.75] (current page.north west) rectangle (current page.south east);%
\end{tikzpicture}%
}}%

\newcommand{\clearbackgroundpicture}{\usebackgroundtemplate{}%
\definecolor{FootColor}{rgb}{0.322,0.322,0.322}%
\definecolor{FootBackgroundColor}{rgb}{1,1,1}%
\setbeamercolor{bottomcolor}{fg=black,bg=gray!15!white}
}

\definecolor{osugray}{HTML}{5e6061}
\definecolor{ccgray}{HTML}{a7b1a6}

\newcommand{\whitebackground}{\setbeamercolor{background canvas}{bg=white,fg=black}\usebeamercolor[fg]{background canvas}}
\newcommand{\blackbackground}{\setbeamercolor{background canvas}{bg=black,fg=white}\usebeamercolor[fg]{background canvas}}

\begin{document}
\whitebackground

%%%%%%%%%%%%%%%%%%%%%%%%%%%%%%%%%%%%%%%%%%%%%%%%%%%%%%%%%%%%%%%%
% TITLE
\clearbackgroundpicture
\begin{frame}[nofills]
  \vspace{5ex}

  \large
  \scaletowidth{\textwidth}{\textbf{The Calculus Knowledge Assessment:}} \\[1ex]


  \scaletowidth{\textwidth}{\parbox{\widthof{an open-source instrument for measuring}}{\vspace{-2ex}\begin{center}an open-source instrument for measuring \\ learning gains in calculus courses\end{center}}}

  \vfill

  \color{osugray}

  \begin{columns}
    \begin{column}{0.36\textwidth}
      \scaletowidth{\textwidth}{\parbox[t]{\widthof{January 10, 2018}}{\textbf{MAA SoTL}\\San Diego, CA\\January 10, 2018}}
    \end{column}
    \hfill
    \begin{column}{0.59\textwidth}
      \scaletowidth{\textwidth}{\parbox[t]{\widthof{Department of Mathematics}}{{\textbf{Jim Fowler}} \\
      {The Ohio State University} \\
      {Department of Mathematics}}}
\end{column}
\end{columns}
  
\end{frame}

% This talk presents data on how the CKA can be used to measure
% learning gains (e.g., in August 2016, a student submitted on average
% 35 correct answers during the pretest while at the end of the
% course, a student submitted on average 47 correct answers)

% and how certain CKA items are highly correlated with success on
% traditional in-class exams.  The fact that the CKA is openly
% licensed and served via the open-source Ximera platform makes it
% easy for other schools to adopt the CKA.


%%%%%%%%%%%%%%%%%%%%%%%%%%%%%%%%%%%%%%%%%%%%%%%%%%%%%%%%%%%%%%%%
% Joint project
\begin{frame}
  \frametitle{Joint work}
  \Large

  This is a joint project with Bart Snapp \\
  \quad and the ``Math Education Forum'' \\
  which considers calculus instruction at Ohio State.
  
\end{frame}

%%%%%%%%%%%%%%%%%%%%%%%%%%%%%%%%%%%%%%%%%%%%%%%%%%%%%%%%%%%%%%%%
% Desiderata
\begin{frame}
  \Large
  \vfill
  
  \textbf{Are our calculus students learning?} \\
  \textbf{And what are they learning?}

  \vfill
  
  To begin to answer these questions, \\
  \quad we administered pretests and posttests \\
  \quad in our calculus courses.
  
\end{frame}

%%%%%%%%%%%%%%%%%%%%%%%%%%%%%%%%%%%%%%%%%%%%%%%%%%%%%%%%%%%%%%%%
% The CKA
\begin{frame}
  \large
  
  The \textbf{Calculus Knowledge Assessment} is \\
  \quad a bank of \textcolor{gray}{currently 221} multiple-choice items.

  \vfill
  
  Problems written by Ohio State faculty\\
  \quad are openly-licensed \textcolor{gray}{under CC--BY--SA.}

  \vfill

  Source at \url{https://github.com/mooculus/concepts}
  
  \vfill

  Delivered initially via optical mark recognition, \\
  \quad but now online via the \textbf{\textsf{XIMERA}} platform.
  
\end{frame}

%%%%%%%%%%%%%%%%%%%%%%%%%%%%%%%%%%%%%%%%%%%%%%%%%%%%%%%%%%%%%%%%
\begin{frame}
  \scaletowidth{\textwidth}{\textbf{Are our calculus students learning?}}
\end{frame}

\begin{frame}
  \frametitle{The nightmare of college calculus instructors}
  \large
  
  Among our students, $>40\%$ have taken an  AP~Calculus course; \\
  \quad and most of the others have taken a course called calculus.

  \vfill
  
  Students with AP~experience score better on our exams \\
  \quad\textcolor{gray}{(77.6\% vs 71.0\%; $p<0.001$; 95\% CI [+4.2\%,+9.0\%])} \\
  so maybe students would pass \textit{even if we did nothing.}

  \vfill
  
  \textbf{Are we adding value?}
  
\end{frame}

\begin{frame}
  \frametitle{Pretest versus posttest}
  \large
  
  \begin{tabular}{@{}l@{ }l@{ }l}
  On the &pretest, &$\approx$ 15.1 correct responses. \\
  On the &posttest, &$\approx$ 19.3 correct responses.
  \end{tabular}

  \vfill

  This is significant \textcolor{gray}{$(p < 0.001)$} with \textcolor{gray}{95\% CI of} \\
  \quad 2.8--5.5 more correct responses on the posttest.

  \vfill

  \textbf{We are adding value.}
  

\end{frame}

\begin{frame}
  \frametitle{Students become more fluent}

  \large

  It takes $\approx$ 48 seconds after displaying a problem \\
  \quad for a student to respond with a correct answer.

  \vfill
  
  \begin{tabular}{@{}l@{ }l@{ }l}
  On the &pretest, &$\approx$ 54.9 seconds/correct response. \\
  On the &posttest, &$\approx$ 46.6 seconds/correct response.
  \end{tabular}

  \vfill

  This is significant \textcolor{gray}{$(p < 0.001)$} with \textcolor{gray}{95\% CI of} \\
  \quad an improvement of 6.0--10.6 seconds/correct response.

\end{frame}


%%%%%%%%%%%%%%%%%%%%%%%%%%%%%%%%%%%%%%%%%%%%%%%%%%%%%%%%%%%%%%%%
\begin{frame}
  \scaletowidth{\textwidth}{\textbf{What are we teaching?}}
\end{frame}

\begin{frame}
  \frametitle{Internal Consistency}
  \large
  
  For the smaller set of problems used Autumn~2017, \\
  \quad Cronbach's $\alpha$ lies in \textcolor{gray}{95\% CI}
  $[0.75,0.78]$.

  \vfill
  
  What is ``dimension'' of the space of calculus problems? \\\quad \textcolor{gray}{This has
  implications for Item Response Theory.}
  
\end{frame}

%%%%%%%%%%%%%%%%%%%%%%%%%%%%%%%%%%%%%%%%%%%%%%%%%%%%%%%%%%%%%%%%
\begin{frame}
  \frametitle{Aligned with traditional exams}

  % This is the score as the mean of correct responses
  
  Posttest score and Final Exam are linearly correlated \textcolor{gray}{($p < 0.001$)} \\
  \quad \textcolor{gray}{with Pearson correlation coefficient $\approx$ 0.361.} \\
  \quad meaning a medium effect size.

  \vfill
  
  Pre-test score and Exam 1 are linearly correlated \textcolor{gray}{($p < 0.001$)} \\
  \quad \textcolor{gray}{with Pearson correlation coefficient $\approx$ 0.315.}

  \vfill
  
  \textbf{But be warned} that Exam 1 and the Final Exam are more correlated \\
  \quad \textcolor{gray}{with Pearson correlation coefficient $\approx$ 0.665.}
  
\end{frame}

%%%%%%%%%%%%%%%%%%%%%%%%%%%%%%%%%%%%%%%%%%%%%%%%%%%%%%%%%%%%%%%%
\begin{frame}

  \prepost{13.0}{40.2}{15.8}{38.5}

\begin{problem}{q144}
  Suppose $f''(x) = f'(x) + f(x)$.\\[1ex]

  If $f'(1) = 0$ and $f(1) = 2$, what is true about $f$?\\[1.5ex]
  
  \choice{$f$ has a local minimum at the point $1$}
  \choice{$f$ has a local maximum at the point $1$}
  \choice{$f$ has an inflection point at $1$}
\end{problem}

\end{frame}

\begin{frame}

  \prepost{15.8}{34.7}{7.5}{30.4}
  
\begin{problem}{q171}
  When $n$ is a large integer, what can be said about \[\int_0^{n \pi} \sin x \, dx?\]
  
    \choice{It is large.}
    \choice{It is close to zero but not exactly zero.}
    \choice{It is exactly zero.}
    \choice{It is very negative.}
\end{problem}
\end{frame}

\begin{frame}
  \frametitle{Some big gains}
  \large
  
  The big gains on q144 \textcolor{gray}{(Suppose $f''(x) = f'(x) + f(x)$\ldots)} \\
  \quad and on q171 \textcolor{gray}{(\ldots$\int_0^{n \pi} \sin x \, dx$\ldots)} \\
  suggest that we are teaching ``calculus'' \\
  \quad and our students are learning ``calculus.''

\end{frame}


\begin{frame}
  \prepost{35.3}{38.1}{-10.1}{15.7}
  
  \begin{problem}{p13}
    Four equally spaced numbers are shown below on a number line.
    \begin{center}
    \begin{tikzpicture}[x=0.19\textwidth]
      \draw[-{>[scale=1.75]}] (-0.05\textwidth,0) -- (0.8\textwidth,0);
      \foreach \x in {0,...,4} {%
        \draw (\x,-.1) -- (\x,.1);
        \node[anchor=north,yshift=-2pt] at (\x,0) {$\x$};
      }
      \draw (0.5,0) node [circle,fill,inner sep=1pt,label=above:$x$](e){};
      \draw (1.5,0) node [circle,fill,inner sep=1pt,label=above:$y$](e){};
      \draw (2.5,0) node [circle,fill,inner sep=1pt,label=above:$z$](e){};
      \draw (3.5,0) node [circle,fill,inner sep=1pt,label=above:$w$](e){};
    \end{tikzpicture}
  \end{center}

  Let $f(t) = \ln t$ and consider $f(x)$ and $f(y)$ and $f(z)$ and
  $f(w)$.  Among the pairs listed below, which pair of numbers is
  farthest apart? \\[1.5ex]
  
    \choice{The pair $f(x)$ and $f(y)$.}
    \choice{The pair $f(y)$ and $f(z)$.}
    \choice{The pair $f(z)$ and $f(w)$.}
    \choice{These pairs are all equally far apart.}
\end{problem}
\end{frame}


\begin{frame}

  \prepost{46.1}{31.3}{-28.3}{-1.2}
  
\begin{problem}{q17}
	Let $f$ and $g$ be two functions satisfying the relationship $f(x) = g(4x)$.

Suppose the point $(2,3)$ is on the graph of $f$.\\[1.5ex]

		\choice{The point $(\frac{1}{2},3)$ is on the graph of $g$}
		\choice{The point $(8, 3)$ is on the graph of $g$}
		\choice{The point $(8, 12)$ is on the graph of $g$}
		\choice{The point $(\frac{1}{2}, 12)$ is on the graph of $g$}
\end{problem}
\end{frame}

\begin{frame}
  \frametitle{Forgetfulness}
  \large

  Problems p13 \textcolor{gray}{(Four equally spaced numbers\ldots)} \\
  \quad and q17 \textcolor{gray}{(\ldots $f(x) = g(4x)$ \ldots)} \\
  are essentially precalculus questions.

  \vfill
  
  We oughtn't be surprised our students forget something\\
  \quad by the end of calculus---\\
  hence efforts like ``stretch calculus.''

\end{frame}


\begin{frame}[nofills]

  \prepost{24.2}{12.7}{-22.2}{0.7}
  
\begin{problem}{q51}
  Suppose $A$ and $a$ differ by at most $2$ and \\
  \quad suppose $B$ and $b$ differ by at most $1$.

  In symbols, this means that $|A - a| < 2$ and $|B - b| < 1$.
  
  By how much could $A/B$ and $a/b$ differ?\vspace{1.5ex}

    \choice{By no more than $1$, meaning $|A/B - a/b| < 1$}
    \choice{By no more than $2$, meaning $|A/B - a/b| < 2$}
    \choice{By no more than a factor of $2$, meaning $\displaystyle\left| \frac{A/B}{a/b} \right| < 2$}
    \choice{By no more than a factor of $1/2$, meaning $\displaystyle\left| \frac{A/B}{a/b} \right| < \frac{1}{2}$}
    \choice{By any amount}
\end{problem}
\end{frame}


%%%%%%%%%%%%%%%%%%%%%%%%%%%%%%%%%%%%%%%%%%%%%%%%%%%%%%%%%%%%%%%%
\begin{frame}
  The CKA is a tool in the fight against \textbf{data balkanization}.

  An advantage publishers have is a common data warehouse.

\end{frame}

%%%%%%%%%%%%%%%%%%%%%%%%%%%%%%%%%%%%%%%%%%%%%%%%%%%%%%%%%%%%%%%%
% THANK YOU
 \clearbackgroundpicture
 \begin{frame}[label=thanks,nofills]
   \vfill
   \begin{center}
   \Huge
    \scalebox{1.5}{\textbf{Thank You}}
   \end{center}
   \vfill
   \vfill
   \ccbysa\hfill\footnotesize\scalebox{0.75}{\textcolor{ccgray}{Licensed for reuse under a Creative Commons BY-NC-SA License}}
   \null
   \null
 \end{frame}

\end{document}
